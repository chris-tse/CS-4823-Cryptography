\documentclass{article}

        \usepackage{fancyhdr}
        \usepackage{extramarks}
        \usepackage{amsmath}
        \usepackage{amssymb}
        \usepackage{amsthm}
        \usepackage{amsfonts}
        \usepackage{tikz}
        \usepackage[plain]{algorithm}
        \usepackage{algpseudocode}
        \usepackage{listings}
        \definecolor{mygreen}{rgb}{0,0.6,0}
        \definecolor{mygray}{rgb}{0.5,0.5,0.5}
        \definecolor{mymauve}{rgb}{0.58,0,0.82}

        \lstdefinelanguage{JavaScript}{
            keywords={typeof, new, true, false, catch, function, return, null, catch, switch, var, if, in, while, do, else, case, break},
            keywordstyle=\color{blue}\bfseries,
            ndkeywords={class, export, boolean, throw, implements, import, this},
            ndkeywordstyle=\color{darkgray}\bfseries,
            identifierstyle=\color{black},
            sensitive=false,
            comment=[l]{//},
            morecomment=[s]{/*}{*/},
            commentstyle=\color{purple}\ttfamily,
            stringstyle=\color{red}\ttfamily,
            morestring=[b]',
            morestring=[b]"
        }
        \lstset{
            language=JavaScript,
            backgroundcolor=\color{white},
            extendedchars=true,
            basicstyle=\footnotesize\ttfamily,
            showstringspaces=false,
            showspaces=false,
            numbers=left,
            numberstyle=\footnotesize,
            numbersep=9pt,
            tabsize=2,
            breaklines=true,
            showtabs=false,
            captionpos=b
        }
        
        \usetikzlibrary{automata,positioning}
        
        %
        % Basic Document Settings
        %
        
        \topmargin=-0.45in
        \evensidemargin=0in
        \oddsidemargin=0in
        \textwidth=6.5in
        \textheight=9.0in
        \headsep=0.25in
        
        \linespread{1.2}
        
        \pagestyle{fancy}
        \lhead{\hmwkAuthorName}
        \chead{\hmwkClass\ \hmwkTitle}
        \rhead{\firstxmark}
        \lfoot{\lastxmark}
        \cfoot{\thepage}
        
        \renewcommand\headrulewidth{0.4pt}
        \renewcommand\footrulewidth{0.4pt}
        
        \setlength\parindent{0pt}
        \setlength\parskip{0.2cm}
        
        %
        % Create Problem Sections
        %
        
        \newcommand{\enterProblemHeader}[1]{
            \nobreak\extramarks{}{Problem \arabic{#1} continued on next page\ldots}\nobreak{}
            \nobreak\extramarks{Problem \arabic{#1} (continued)}{Problem \arabic{#1} continued on next page\ldots}\nobreak{}
        }
        
        \newcommand{\exitProblemHeader}[1]{
            \nobreak\extramarks{Problem \arabic{#1} (continued)}{Problem \arabic{#1} continued on next page\ldots}\nobreak{}
            \stepcounter{#1}
            \nobreak\extramarks{Problem \arabic{#1}}{}\nobreak{}
        }
        
        \setcounter{secnumdepth}{0}
        \newcounter{partCounter}
        \newcounter{homeworkProblemCounter}
        \setcounter{homeworkProblemCounter}{1}
        \nobreak\extramarks{Problem \arabic{homeworkProblemCounter}}{}\nobreak{}
        
        %
        % Homework Problem Environment
        %
        % This environment takes an optional argument. When given, it will adjust the
        % problem counter. This is useful for when the problems given for your
        % assignment aren't sequential. See the last 3 problems of this template for an
        % example.
        %
        \newenvironment{homeworkProblem}[1][-1]{
            \ifnum#1>0
                \setcounter{homeworkProblemCounter}{#1}
            \fi
            \section{Problem \arabic{homeworkProblemCounter}}
            \setcounter{partCounter}{1}
            \enterProblemHeader{homeworkProblemCounter}
        }{
            \exitProblemHeader{homeworkProblemCounter}
        }
        
        %
        % Homework Details
        %   - Title
        %   - Due date
        %   - Class
        %   - Author
        %
        
        \newcommand{\hmwkTitle}{Homework\ \#7}
        \newcommand{\hmwkDueDate}{March 9, 2018}
        \newcommand{\hmwkClass}{CS 4823}
        \newcommand{\hmwkAuthorName}{\textbf{Christopher Tse}}
        
        %
        % Title Page
        %
        
        \title{
            \vspace{2in}
            \textmd{\textbf{\hmwkClass:\ \hmwkTitle}}\\
            \normalsize\vspace{0.1in}\small{Due\ on\ \hmwkDueDate}\\
            \vspace{3in}
        }
        
        \author{\hmwkAuthorName}
        \date{}
        
        \renewcommand{\part}[1]{\textbf{\large Part \Alph{partCounter}}\stepcounter{partCounter}\\}
        
        %
        % Various Helper Commands
        %
        
        % Useful for algorithms
        \newcommand{\alg}[1]{\textsc{\bfseries \footnotesize #1}}
        
        % For derivatives
        \newcommand{\deriv}[1]{\frac{\mathrm{d}}{\mathrm{d}x} (#1)}
        
        % For partial derivatives
        \newcommand{\pderiv}[2]{\frac{\partial}{\partial #1} (#2)}
        
        % Integral dx
        \newcommand{\dx}{\mathrm{d}x}
        
        % Alias for the Solution section header
        \newcommand{\solution}{\textbf{\large Solution}}
        
        % Probability commands: Expectation, Variance, Covariance, Bias
        \newcommand{\E}{\mathrm{E}}
        \newcommand{\Var}{\mathrm{Var}}
        \newcommand{\Cov}{\mathrm{Cov}}
        \newcommand{\Bias}{\mathrm{Bias}}
        \newcommand{\Z}{\mathbb{Z}}
        
        \begin{document}
        
        \maketitle
        
        \pagebreak
        
        \begin{homeworkProblem}
            Let $w$ be a string over $\lbrace A, B, ... , Z \rbrace$. Choose two Caesar cipher keys $k_1$ and $k_2$. Encrypt the symbols of $w$ having odd index using $k_1$ and those having even index using $k_2$. Then reverse the order of the encrypted string.
            
            Show that the above procedure defines a cryptosystem. 
            
            \textbf{Solution}
            
            A cryptosystem is defined as a tuple $(P, C, K, E, D)$ such that $P$ is a set for the plaintext space, $C$ is a set for the ciphertext space, $K$ is a set for the key space, $E$ is a family of encryption functions, and $D$ is a family of decryption functions. 
            
            The above system is a cyphersystem since there is a plaintext $w$ in the plaintext space $\lbrace A, B, ... , Z \rbrace$ as described. There is also a final encrypted string which is the ciphertext, which corresponds to $C$. We have a key space $K$ which is defined by $k_1$ and $k_2$. Using this key, we can determine the encryption and decryption functions for $E$ and $D$ like we would for a normal Caesar cipher.
            
            Determine the plaintext space, the ciphertext space, and the key space.
            
            \textbf{Solution}
            
            The plaintext and ciphertext space are the same as described in the problem, $\lbrace A, B, ... , Z \rbrace$.
            
            The keyspace for both $k_1$ and $k_2$ are any monic degree 1 polynomial since the keys are always in the form of $c = p + n$ for some integer $n$ in a Caesar cipher. 
            
        \end{homeworkProblem}
        
        % \pagebreak
        
        \begin{homeworkProblem}
            What is the maximum number of different encryption functions of a block cipher over the alphabet $\lbrace 0, 1 \rbrace$ with block length $n$?
            
            \textbf{Solution}
            
            Since block ciphers are permutations, we can find the number of different encryption functions by finding the number of unique permutations in some block. Let $n$ be the block length, $n_0$ be the number of 0's in the block and $n_1$ be the number of 1's in the block. We then get the number of encryption keys $n_e$ with \[ n_e = \frac{n!}{n_0!n_1!} \]
            
        \end{homeworkProblem}
        
        \pagebreak
        
        \begin{homeworkProblem}
            Read the page on frequency analysis and write a program to calculate the frequencies of English letters (case-insensitive) in the section "History and Usage" (not including the title of the section and the text in the figures).
            
            \textbf{Solution}
            
            \begin{lstlisting}[language=javascript]
// frequency.js
if (process.argv.length < 3) {
    console.log('Usage: node frequency.js FILENAME');
    process.exit(1);
}
    
const fs = require('fs');
const filename = process.argv[2];

// read in input file
fs.readFile(filename, 'utf8', (err, data) => {
    if (err) throw err;

    // make the text all lowercase and remove non-alphabet characters then split into array of individual characters
    let textArr = data.toLowerCase().replace(/[^a-z]/g, '').split('');

    // reduce the array of characters, incrementing count of character if it exists in the object or initializing to 1 if it does not exist yet
    let result = textArr.reduce( (acc, curr) => {
        acc[curr] ? acc[curr]++ : acc[curr] = 1;
        return acc;
    }, {});

    console.log(result);
});             
            \end{lstlisting}
            
            \begin{verbatim}
$ node frequency.js input.txt
            \end{verbatim}
        \end{homeworkProblem}
        
        \pagebreak
        
        \begin{homeworkProblem}
            Suppose that we use Caesar cipher with multiplication over $\Z/26\Z$ (i.e. affine cipher): \[ c = 11 p + 5. \]
            
            Can you find the formula for decryption? 
            
            \textbf{Solution}
            
            To find the decryption formula we simply perform the inverse operations. We first subtract 5 from the cipher text then multiply by the inverse of the multiplicative factor 11, which is 19. Therefore: 
            \[ p = 19 (c - 5) \]
            
            What is the ciphertext for "TEXAS"? 
            
            \textbf{Solution}
            
            "GXYFV"
            
            What is the plaintext for "OKLAHOMA" if we treat it as ciphertext?
            
            \textbf{Solution}
            
            "PRKJMPDJ"
        \end{homeworkProblem}
\end{document}