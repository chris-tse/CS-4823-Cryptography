\documentclass{article}

        \usepackage{fancyhdr}
        \usepackage{extramarks}
        \usepackage{amsmath}
        \usepackage{amssymb}
        \usepackage{amsthm}
        \usepackage{amsfonts}
        \usepackage{tikz}
        \usepackage[plain]{algorithm}
        \usepackage{algpseudocode}
        
        \usetikzlibrary{automata,positioning}
        
        %
        % Basic Document Settings
        %
        
        \topmargin=-0.45in
        \evensidemargin=0in
        \oddsidemargin=0in
        \textwidth=6.5in
        \textheight=9.0in
        \headsep=0.25in
        
        \linespread{1.2}
        
        \pagestyle{fancy}
        \lhead{\hmwkAuthorName}
        \chead{\hmwkClass\ \hmwkTitle}
        \rhead{\firstxmark}
        \lfoot{\lastxmark}
        \cfoot{\thepage}
        
        \renewcommand\headrulewidth{0.4pt}
        \renewcommand\footrulewidth{0.4pt}
        
        \setlength\parindent{0pt}
        \setlength\parskip{0.2cm}
        
        %
        % Create Problem Sections
        %
        
        \newcommand{\enterProblemHeader}[1]{
            \nobreak\extramarks{}{Problem \arabic{#1} continued on next page\ldots}\nobreak{}
            \nobreak\extramarks{Problem \arabic{#1} (continued)}{Problem \arabic{#1} continued on next page\ldots}\nobreak{}
        }
        
        \newcommand{\exitProblemHeader}[1]{
            \nobreak\extramarks{Problem \arabic{#1} (continued)}{Problem \arabic{#1} continued on next page\ldots}\nobreak{}
            \stepcounter{#1}
            \nobreak\extramarks{Problem \arabic{#1}}{}\nobreak{}
        }
        
        \setcounter{secnumdepth}{0}
        \newcounter{partCounter}
        \newcounter{homeworkProblemCounter}
        \setcounter{homeworkProblemCounter}{1}
        \nobreak\extramarks{Problem \arabic{homeworkProblemCounter}}{}\nobreak{}
        
        %
        % Homework Problem Environment
        %
        % This environment takes an optional argument. When given, it will adjust the
        % problem counter. This is useful for when the problems given for your
        % assignment aren't sequential. See the last 3 problems of this template for an
        % example.
        %
        \newenvironment{homeworkProblem}[1][-1]{
            \ifnum#1>0
                \setcounter{homeworkProblemCounter}{#1}
            \fi
            \section{Problem \arabic{homeworkProblemCounter}}
            \setcounter{partCounter}{1}
            \enterProblemHeader{homeworkProblemCounter}
        }{
            \exitProblemHeader{homeworkProblemCounter}
        }
        
        %
        % Homework Details
        %   - Title
        %   - Due date
        %   - Class
        %   - Author
        %
        
        \newcommand{\hmwkTitle}{Homework\ \#4}
        \newcommand{\hmwkDueDate}{February 16, 2018}
        \newcommand{\hmwkClass}{CS 4823}
        \newcommand{\hmwkAuthorName}{\textbf{Christopher Tse}}
        
        %
        % Title Page
        %
        
        \title{
            \vspace{2in}
            \textmd{\textbf{\hmwkClass:\ \hmwkTitle}}\\
            \normalsize\vspace{0.1in}\small{Due\ on\ \hmwkDueDate}\\
            \vspace{3in}
        }
        
        \author{\hmwkAuthorName}
        \date{}
        
        \renewcommand{\part}[1]{\textbf{\large Part \Alph{partCounter}}\stepcounter{partCounter}\\}
        
        %
        % Various Helper Commands
        %
        
        % Useful for algorithms
        \newcommand{\alg}[1]{\textsc{\bfseries \footnotesize #1}}
        
        % For derivatives
        \newcommand{\deriv}[1]{\frac{\mathrm{d}}{\mathrm{d}x} (#1)}
        
        % For partial derivatives
        \newcommand{\pderiv}[2]{\frac{\partial}{\partial #1} (#2)}
        
        % Integral dx
        \newcommand{\dx}{\mathrm{d}x}
        
        % Alias for the Solution section header
        \newcommand{\solution}{\textbf{\large Solution}}
        
        % Probability commands: Expectation, Variance, Covariance, Bias
        \newcommand{\E}{\mathrm{E}}
        \newcommand{\Var}{\mathrm{Var}}
        \newcommand{\Cov}{\mathrm{Cov}}
        \newcommand{\Bias}{\mathrm{Bias}}
        \newcommand{\Z}{\mathbb{Z}}
        
        \begin{document}
        
        \maketitle
        
        \pagebreak
        
        \begin{homeworkProblem}
            Compute the subgroup generated by $2 + 17\Z$ in $(\Z/17\Z)$*.
            
            \textbf{Solution}
            
            $2^0\ mod\ 17 = 1$\\
            $2^1\ mod\ 17 = 2$\\
            $2^2\ mod\ 17 = 4$\\
            $2^3\ mod\ 17 = 8$\\
            $2^4\ mod\ 17 = 16$\\
            $2^5\ mod\ 17 = 15$\\
            $2^6\ mod\ 17 = 13$\\
            $2^7\ mod\ 17 = 9$\\
            $2^8\ mod\ 17 = 1$ 
            
            Therefore, the subgroup is $1 \rightarrow 2 \rightarrow 4 \rightarrow 8 \rightarrow 16 \rightarrow 15 \rightarrow 13 \rightarrow 9 \rightarrow 1$
            
        \end{homeworkProblem}
        
        \begin{homeworkProblem}
            Determine the order of all the elements in $(\Z/15\Z)$*.
            
            \textbf{Solution}
            
            Unit group of $(\Z/15\Z)$* is $\lbrace 1, 2, 4, 7, 8, 11, 13, 14 \rbrace$
            
            $1^1 = 1\ mod\ 15  = 1 \Rightarrow$ order 1\\
            $2^4 = 16\ mod\ 15 = 1 \Rightarrow$ order 4\\
            $4^2 = 16\ mod\ 15 = 1 \Rightarrow$ order 2\\
            $7^4 = 2401\ mod\ 15 = 1 \Rightarrow$ order 4\\
            $8^4 = 4096\ mod\ 15 = 1 \Rightarrow$ order 4\\
            $11^2 = 121\ mod\ 15 = 1 \Rightarrow$ order 2\\
            $13^4 = 28561\ mod\ 15 = 1 \Rightarrow$ order 4\\
            $14^2 = 196\ mod\ 15 = 1 \Rightarrow$ order 2\\
            
        \end{homeworkProblem}
        
        \pagebreak
        
        \begin{homeworkProblem}
            In Sage, after initiation:
            \begin{verbatim}
    sage: R = Integers(2387591645982364564382654564856487)
    sage: a = 209734827465248974582964584
    sage: b = 834574895748236582648752475485
            \end{verbatim}
            
            If we run \texttt{sage: R(a)\^{}b} we get the answer 2341670245383644195337830861352166. However, if we run \texttt{sage: R(a\^{}b)} we get "RuntimeError". Explain why by estimating how much disk space (in GBytes) is needed to store the result of $a^b$ in binary.
            
            \textbf{Solution}
            
            The number of bits required to store $a^b$ in binary is $\log_2(a^b)$. While normally this logarithm is too large to compute, we can use the logarithm exponent rule. 
            
            \begin{align*}
                \log_2(a^b) &= b\log_2(a)\\
                &= 834574895748236582648752475485\log_2(209734827465248974582964584)\\
                &= 7.297 \times 10^{31}
            \end{align*}
            
            The number of bits required to store $a^b$ is approximately $7.297 \times 10^{31}$ bits, or approximately $9.122 \times 10^{21}$ gigabytes.
        \end{homeworkProblem}
        
        % \pagebreak
        
        \begin{homeworkProblem}
            Prove that RSA-1024 is a composite number using the Fermat Little Theorem with $a$ = your id number.
            
            \textbf{Solution: Proof by contradiction}
            
            Assume $p$ = RSA-1024 = 135066410865995223349603216278805969938881475\\60566702752448514385152651060485953383394028715057190944179820728216\\44715513736804197039641917430464965892742562393410208643832021103729\\58725762358509643110564073501508187510676594629205563685529475213500\\852879416377328533906109750544334999811150056977236890927563
            
            According to Fermat's Little Theorem, if $p$ is prime and $a \nmid p$, then $a^{p-1} \equiv 1 (mod\ p)$, or $a^p \equiv a (mod\ p)$.
            
            Since $a^p\ mod\ p$ is too large to compute, we can first take $a\ mod\ p$ then raise it to the $p$-th power. In this case, $a\ mod\ p = a$. According to Fermat's Little Theorem then,
            
            \begin{align*}
                (a\ mod\ p)^p = a\\
                a^p \neq a
            \end{align*}
            
            Since $a^p \neq a$, Fermat's Little Theorem does not hold true for $p$, therefore $p$ is not a prime and is composite.
        \end{homeworkProblem}
\end{document}