\documentclass{article}

        \usepackage{fancyhdr}
        \usepackage{extramarks}
        \usepackage{amsmath}
        \usepackage{amsthm}
        \usepackage{amsfonts}
        \usepackage{tikz}
        \usepackage[plain]{algorithm}
        \usepackage{algpseudocode}
        
        \usetikzlibrary{automata,positioning}
        
        %
        % Basic Document Settings
        %
        
        \topmargin=-0.45in
        \evensidemargin=0in
        \oddsidemargin=0in
        \textwidth=6.5in
        \textheight=9.0in
        \headsep=0.25in
        
        \linespread{1.2}
        
        \pagestyle{fancy}
        \lhead{\hmwkAuthorName}
        \chead{\hmwkClass\ \hmwkTitle}
        \rhead{\firstxmark}
        \lfoot{\lastxmark}
        \cfoot{\thepage}
        
        \renewcommand\headrulewidth{0.4pt}
        \renewcommand\footrulewidth{0.4pt}
        
        \setlength\parindent{0pt}
        \setlength\parskip{0.2cm}
        
        %
        % Create Problem Sections
        %
        
        \newcommand{\enterProblemHeader}[1]{
            \nobreak\extramarks{}{Problem \arabic{#1} continued on next page\ldots}\nobreak{}
            \nobreak\extramarks{Problem \arabic{#1} (continued)}{Problem \arabic{#1} continued on next page\ldots}\nobreak{}
        }
        
        \newcommand{\exitProblemHeader}[1]{
            \nobreak\extramarks{Problem \arabic{#1} (continued)}{Problem \arabic{#1} continued on next page\ldots}\nobreak{}
            \stepcounter{#1}
            \nobreak\extramarks{Problem \arabic{#1}}{}\nobreak{}
        }
        
        \setcounter{secnumdepth}{0}
        \newcounter{partCounter}
        \newcounter{homeworkProblemCounter}
        \setcounter{homeworkProblemCounter}{1}
        \nobreak\extramarks{Problem \arabic{homeworkProblemCounter}}{}\nobreak{}
        
        %
        % Homework Problem Environment
        %
        % This environment takes an optional argument. When given, it will adjust the
        % problem counter. This is useful for when the problems given for your
        % assignment aren't sequential. See the last 3 problems of this template for an
        % example.
        %
        \newenvironment{homeworkProblem}[1][-1]{
            \ifnum#1>0
                \setcounter{homeworkProblemCounter}{#1}
            \fi
            \section{Problem \arabic{homeworkProblemCounter}}
            \setcounter{partCounter}{1}
            \enterProblemHeader{homeworkProblemCounter}
        }{
            \exitProblemHeader{homeworkProblemCounter}
        }
        
        %
        % Homework Details
        %   - Title
        %   - Due date
        %   - Class
        %   - Author
        %
        
        \newcommand{\hmwkTitle}{Homework\ \#3}
        \newcommand{\hmwkDueDate}{February 9, 2018}
        \newcommand{\hmwkClass}{CS 4823}
        \newcommand{\hmwkAuthorName}{\textbf{Christopher Tse}}
        
        %
        % Title Page
        %
        
        \title{
            \vspace{2in}
            \textmd{\textbf{\hmwkClass:\ \hmwkTitle}}\\
            \normalsize\vspace{0.1in}\small{Due\ on\ \hmwkDueDate}\\
            \vspace{3in}
        }
        
        \author{\hmwkAuthorName}
        \date{}
        
        \renewcommand{\part}[1]{\textbf{\large Part \Alph{partCounter}}\stepcounter{partCounter}\\}
        
        %
        % Various Helper Commands
        %
        
        % Useful for algorithms
        \newcommand{\alg}[1]{\textsc{\bfseries \footnotesize #1}}
        
        % For derivatives
        \newcommand{\deriv}[1]{\frac{\mathrm{d}}{\mathrm{d}x} (#1)}
        
        % For partial derivatives
        \newcommand{\pderiv}[2]{\frac{\partial}{\partial #1} (#2)}
        
        % Integral dx
        \newcommand{\dx}{\mathrm{d}x}
        
        % Alias for the Solution section header
        \newcommand{\solution}{\textbf{\large Solution}}
        
        % Probability commands: Expectation, Variance, Covariance, Bias
        \newcommand{\E}{\mathrm{E}}
        \newcommand{\Var}{\mathrm{Var}}
        \newcommand{\Cov}{\mathrm{Cov}}
        \newcommand{\Bias}{\mathrm{Bias}}
        
        \begin{document}
        
        \maketitle
        
        \pagebreak
        
        \begin{homeworkProblem}
            Solve $122x \equiv 3(mod\ 343)$. Show step-by-step calculations.
            
            \textbf{Solution}
        
            \textbf{Finding GCD}\\
            Using Euclidean Division Algorithm: 
            \begin{align*}
                343 &= 122(2) + 99\\
                122 &= 99(1) + 23\\
                99 &= 23(4) + 7\\
                23 &= 7(3) + 2\\
                7 &= 2(3) + 1\\
            \end{align*}
            Follow with Extended Euclidean Algorithm:
            \begin{align*}
                1 &= 7 + 2(-3)\\
                &= 7 + (23 + 7(-3))(-3) = 7(10) + 23(-3)\\
                &= (99 + 23(-4))(10) + 23(-3) = 99(10) + 23(-43)\\
                &= 99(10) + (122 + 99(-1))(-43) = 99(53) + 122(-43)\\
                &= (343 + 122(-2))(53) + 122(-43)\\
                1 &= 343(53) + 122(-149)
            \end{align*}
            In the case of $122x \equiv 1 (mod\ 343)$ we have $x = -149 \equiv 194 (mod\ 343)$, so for $122x \equiv 3 (mod\ 343)$ we have $x = 194 * 3 (mod\ 343) = 239$
        \end{homeworkProblem}
        
        \pagebreak
        
        \begin{homeworkProblem}
            \textbf{a)} Is your ID number invertible modulo $m = 2^{64}$?   
            
            ID = 112971666, since $gcd(ID, 2^{64}) = 2$, there is no inverse.
            
            \textbf{b)} Let $a$ be the least integer that is no less than your ID number and is invertible mod m. Use Sage xgcd to find the inverse of $a\ mod\ m$.
            \begin{verbatim}
    sage: a = 112971667
    sage: g,x,y = xgcd(a, 2**64)
    sage: print(g,x,y)
    (1, -3514066771570022757, 21520870) 
            \end{verbatim}

            From the above output from Sage, we can see that $1 = a(-3514066771570022757) + 2^{64}(21520870)$. Therefore, inverse of $a$ is $-3514066771570022757$.
            
            \textbf{c)} In a C++ program, assume that there is a variable $x$ with type "unsigned long int" (64bits), and the product of $a$ and $x$ is 2018, what is x?
            \begin{verbatim}
    // main.cpp
    #include <iostream>

    using namespace std;
    
    int main() {
        unsigned long int a = 112971666;
        cout << a / 2018 << endl;
        return 0;
    }
    
    // bash output
    $ make main
    c++     main.cpp   -o main
    $ ./main
    55981
            \end{verbatim}
        
        \end{homeworkProblem}
        
        \pagebreak
        
        \begin{homeworkProblem}
            Determine the unit group and the zero divisors of the ring $\mathbb{Z}/16\mathbb{Z}$.
            
            \textbf{Unit Group}
            
            Unit group is the set of elements in $16\mathbb{Z}$ where $a_i \in \mathbb{Z}$ and $gcd(a_i, 16) = 1$. Therefore: $\lbrace 1, 3, 5, 7, 9, 11, 13, 15 \rbrace$
            
            \textbf{Zero Divisors}
            
            Zero divisors is the set of elements in $16\mathbb{Z}$ where $a_i \in \mathbb{Z}$ and $gcd(a_i, 16) \neq 1$. Therefore: $\lbrace 2, 4, 6, 8, 10, 12, 14 \rbrace$
        \end{homeworkProblem}
        
        \begin{homeworkProblem}
            Determine the unit group and the zero divisors of the ring $\mathbb{Z}/15\mathbb{Z}$.
            
            \textbf{Unit Group}
            
            Unit group is the set of elements in $15\mathbb{Z}$ where $a_i \in \mathbb{Z}$ and $gcd(a_i, 15) = 1$. Therefore: $\lbrace 1, 2, 4, 7, 8, 11, 13, 14 \rbrace$
            
            \textbf{Zero Divisors}
            
            Zero divisors is the set of elements in $15\mathbb{Z}$ where $a_i \in \mathbb{Z}$ and $gcd(a_i, 15) \neq 1$. Therefore: $\lbrace 3, 5, 6, 9, 10, 12 \rbrace$
        \end{homeworkProblem}
        
        \end{document}