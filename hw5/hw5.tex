\documentclass{article}

        \usepackage{fancyhdr}
        \usepackage{extramarks}
        \usepackage{amsmath}
        \usepackage{amssymb}
        \usepackage{amsthm}
        \usepackage{amsfonts}
        \usepackage{tikz}
        \usepackage[plain]{algorithm}
        \usepackage{algpseudocode}
        
        \usetikzlibrary{automata,positioning}
        
        %
        % Basic Document Settings
        %
        
        \topmargin=-0.45in
        \evensidemargin=0in
        \oddsidemargin=0in
        \textwidth=6.5in
        \textheight=9.0in
        \headsep=0.25in
        
        \linespread{1.2}
        
        \pagestyle{fancy}
        \lhead{\hmwkAuthorName}
        \chead{\hmwkClass\ \hmwkTitle}
        \rhead{\firstxmark}
        \lfoot{\lastxmark}
        \cfoot{\thepage}
        
        \renewcommand\headrulewidth{0.4pt}
        \renewcommand\footrulewidth{0.4pt}
        
        \setlength\parindent{0pt}
        \setlength\parskip{0.2cm}
        
        %
        % Create Problem Sections
        %
        
        \newcommand{\enterProblemHeader}[1]{
            \nobreak\extramarks{}{Problem \arabic{#1} continued on next page\ldots}\nobreak{}
            \nobreak\extramarks{Problem \arabic{#1} (continued)}{Problem \arabic{#1} continued on next page\ldots}\nobreak{}
        }
        
        \newcommand{\exitProblemHeader}[1]{
            \nobreak\extramarks{Problem \arabic{#1} (continued)}{Problem \arabic{#1} continued on next page\ldots}\nobreak{}
            \stepcounter{#1}
            \nobreak\extramarks{Problem \arabic{#1}}{}\nobreak{}
        }
        
        \setcounter{secnumdepth}{0}
        \newcounter{partCounter}
        \newcounter{homeworkProblemCounter}
        \setcounter{homeworkProblemCounter}{1}
        \nobreak\extramarks{Problem \arabic{homeworkProblemCounter}}{}\nobreak{}
        
        %
        % Homework Problem Environment
        %
        % This environment takes an optional argument. When given, it will adjust the
        % problem counter. This is useful for when the problems given for your
        % assignment aren't sequential. See the last 3 problems of this template for an
        % example.
        %
        \newenvironment{homeworkProblem}[1][-1]{
            \ifnum#1>0
                \setcounter{homeworkProblemCounter}{#1}
            \fi
            \section{Problem \arabic{homeworkProblemCounter}}
            \setcounter{partCounter}{1}
            \enterProblemHeader{homeworkProblemCounter}
        }{
            \exitProblemHeader{homeworkProblemCounter}
        }
        
        %
        % Homework Details
        %   - Title
        %   - Due date
        %   - Class
        %   - Author
        %
        
        \newcommand{\hmwkTitle}{Homework\ \#5}
        \newcommand{\hmwkDueDate}{February 23, 2018}
        \newcommand{\hmwkClass}{CS 4823}
        \newcommand{\hmwkAuthorName}{\textbf{Christopher Tse}}
        
        %
        % Title Page
        %
        
        \title{
            \vspace{2in}
            \textmd{\textbf{\hmwkClass:\ \hmwkTitle}}\\
            \normalsize\vspace{0.1in}\small{Due\ on\ \hmwkDueDate}\\
            \vspace{3in}
        }
        
        \author{\hmwkAuthorName}
        \date{}
        
        \renewcommand{\part}[1]{\textbf{\large Part \Alph{partCounter}}\stepcounter{partCounter}\\}
        
        %
        % Various Helper Commands
        %
        
        % Useful for algorithms
        \newcommand{\alg}[1]{\textsc{\bfseries \footnotesize #1}}
        
        % For derivatives
        \newcommand{\deriv}[1]{\frac{\mathrm{d}}{\mathrm{d}x} (#1)}
        
        % For partial derivatives
        \newcommand{\pderiv}[2]{\frac{\partial}{\partial #1} (#2)}
        
        % Integral dx
        \newcommand{\dx}{\mathrm{d}x}
        
        % Alias for the Solution section header
        \newcommand{\solution}{\textbf{\large Solution}}
        
        % Probability commands: Expectation, Variance, Covariance, Bias
        \newcommand{\E}{\mathrm{E}}
        \newcommand{\Var}{\mathrm{Var}}
        \newcommand{\Cov}{\mathrm{Cov}}
        \newcommand{\Bias}{\mathrm{Bias}}
        \newcommand{\Z}{\mathbb{Z}}
        
        \begin{document}
        
        \maketitle
        
        \pagebreak
        
        \begin{homeworkProblem}
            Let $id$ be your student ID number. Solve the simultaneous congruences:
            $$
            \begin{cases}
                x \equiv 2\mod297359071\\
                x \equiv 2\mod837582957839\\
                x \equiv 4\mod id\\                
            \end{cases}
            $$
            
            \textbf{Solution}
            
            Solving the simultaneous congruence using Chinese Remainder Theorem:
            
            $m_1 = 297359071,\ m_2 = 837582957839,\ m_3 = id = 112971666$\\
            $M = m_1 \cdot m_2 \cdot m_3 = 28137049647881671915457739954$\\
            $M_1 = \frac{M}{m_1} = 94623142160279589774$\\
            $M_2 = \frac{M}{m_2} = 33593149651082286$\\
            $M_3 = \frac{M}{m_3} = 249062890228437207569$\\
            
            Find integers $y_i$ such that $y_iM_i \equiv 1\mod m_i$. For each case, the Extended Euclidean Algorithm can be used. Using the \texttt{xgcd} function in Sage:
            
            $y_1M_1 \equiv 1\mod m_1$\\
            $94623142160279589774y_1 \equiv 1 \mod 297359071$\\
            $y_1 = 16501321$\\
            \\
            $y_2M_2 \equiv 1\mod m_2$\\
            $335931496510822864y_2 \equiv 1 \mod 837582957839$\\
            $y_2 = -358052305891$\\
            \\
            $y_3M_3 \equiv 1\mod m_3$\\
            $249062890228437207569y_3 \equiv 1 \mod 112971666$\\
            $y_3 = 42024317$
            
            For the simultaneous congruence, $x$ can be expressed as: 
            \begin{align*}
                x &= \sum_{i=1}^{3}a_iy_iM_i\\
                &= (2 \cdot 16501321 \cdot 94623142160279589774)+(2 \cdot  -358052305891 \cdot 33593149651082286)\\
                &+(4 \cdot 42024317 \cdot 249062890228437207569)\\
                &= -20933395454729205022469678854
            \end{align*}
        \end{homeworkProblem}
        
        \pagebreak
        
        \begin{homeworkProblem}
            \textbf{Part 1}\\
            Let $id$ be your student ID number, $p$ be the prime number 93935935937584760927320853927657, and $q$ be the prime number 20395358947549853439147504976967820947509174847. Find an integer $x$ such that\\$x^{37} \equiv id (\mod n)$, where $n = p \cdot q$.
            
            \textbf{Solution}
            \begin{align*}
                x^{37} &\equiv id \mod n\\
                x^{37} &\equiv 112971666 \mod n\\        x^{37} &\equiv \begin{cases}
                    112971666 \mod p\\
                    112971666 \mod q
                \end{cases}
            \end{align*}
            
            We can reduce the exponents using Euler's Phi function then taking the modulus, but since $p$ and $q$ are so big, the resulting powers are still 37.
            
            This gives us the result:
            
            $$x = \begin{cases}
                57156593804643713070162779699449\\
                17296737745793791981935423565575416285014857800
            \end{cases}
            $$
            
            \textbf{Part 2}\\
            If you do not know the factorization of $n$, can you find $x$ quickly?
            
            No
        \end{homeworkProblem}
        
        \pagebreak
        
        \begin{homeworkProblem}
            Find all the positive integers $m$ such that $(\Z/m\Z)^*$ has four elements.
            
            \textbf{Solution}
            
            Using Euler's Phi Function, we must find some integers $m$ where $m = p^n$ such that $p$ is prime and $n$ is natural. $\phi(p^n) = p^n - p^{n-1}$. Therefore:
            
            \begin{align*}
                4 &= p^n - p^{n-1} \\
                4 &= (p - 1)p^{n-1}\\
            \end{align*}
            
            We can solve for $p$ using the factorizations of 4. The factorizations of 4 are $1 \cdot 4$ and $2 \cdot 2$. Therefore: 
            \begin{align*}
                p &= \begin{cases}
                    2 &\mbox {where}\ n = 3 \\
                    5 &\mbox {where}\ n = 1 \\
                \end{cases} \\
                &\Downarrow \\
                m &= 2^3, 5^1 \\
                m &= 5, 8 
            \end{align*}
            
            Using the factorization of 4 into $1 \cdot 4$ and $2 \cdot 2$ we can also use the multiplicative property of Euler's Phi function. Assume some $x$ and $y$ such that $m = xy$:
            
            $$
                \phi(m) = \phi(x)\phi(y)  
            $$
            
            Using the factors 1 and 4:\\
            $\phi(x) = 1,\ \phi(y) = 4$\\
            $x = 2$\\
            
            Using 5 from our answer above since $\phi(y) = \phi(m) = 4$, we get:\\
            $\phi(m) = \phi(2)\phi(5)$\\
            $m = 10$\\
            (8 is ignored since $\phi(2)\phi(8) = \phi(16) \neq 4$)
            
            However, the factors can be further split into $1 \cdot 2 \cdot 2$:\\
            
            $\phi(x) = 1,\ \phi(y_1) = 2,\ \phi(y_2) = 2$\\
            $x = 2,\ y_1 = 3,\ y_2 = 3$\\
            $\phi(m) = \phi(2)\phi(3)\phi(6)$\\
            $m = 12$
            
            Putting them together, we get $m = 5, 8, 10, 12$
            
        \end{homeworkProblem}
        
        \begin{homeworkProblem}
            Calculate by hand $31^{30^{45}} \mod 35$ using Chinese Remainder Theorem
            
            \textbf{Solution}
            First we split the modulus 35 into its prime factors 5 and 7:
            $$
            \begin{cases}
                31^{30^{45}} \mod 5\\
                31^{30^{45}} \mod 7
            \end{cases}
            $$
            
            To reduce the exponent we take the modulus of the phi function of each factor:
            
            \begin{align*}
                \phi(5) &= 4\\
                \phi(7) &= 6
            \end{align*}
            
            We then obtain:
            \begin{align*}
                30^{45} \mod 4\\
                30^{45} \mod 6\\
            \end{align*}
            
            Since $30 \mod 6 = 0$ then $30^{45} \mod 6 = 0$
            We can use fast modular exponentiation to calculate these:
            
            \begin{align*}
                30^{1+4+8+32} \mod 4 = (30 \cdot 30^4 \cdot 30^8 \cdot 30^{32}) \mod 4
            \end{align*}
            
            \begin{align*}
                30 \mod 4 &= 2\\
                \\
                30^4 \mod 4 &= (30 \mod 4)(30 \mod 4)(30 \mod 4)(30 \mod 4) \mod 4\\
                30^2 \mod 4 &= (2 \cdot 2 \cdot 2 \cdot 2) \mod 4\\
                30^2 \mod 4 &= 16 \mod 4 = 0
            \end{align*}
            
            We do not have to calculate the rest since we multiply the rest of the results. Since one is 0, the end result will be 0. This results in:
            
            \begin{align*}
                30^{45} \equiv 0 \mod 4 \\
                30^{45} \equiv 0 \mod 6\\
            \end{align*}
            
            Replacing this into our original split simultaneous congruence, we get
            
            $$
            \begin{cases}
                x \equiv 31^{30^{45}} \equiv 31^0 \equiv 1 \mod 5\\
                x \equiv 31^{30^{45}} \equiv 31^0 \equiv 1 \mod 7
            \end{cases}
            $$
            
            We can now solve the resulting simultaneous congruence:
            
            $$
            \begin{cases}
                x \equiv 1 \mod 5\\
                x \equiv 1 \mod 7
            \end{cases}
            $$
            
            $m_1 = 5, m_2 = 7$\\
            $M = m_1 \cdot m_2 = 35$\\
            $M_1 = \frac{M}{m_1} = 7$\\
            $M_2 = \frac{M}{m_2} = 5$\\
            
            Find integers $y_i$ such that $y_iM_i \equiv 1 \mod m_i$:
            
            $y_1M_1 \equiv 1 \mod m_1$\\
            $7y_1 \equiv 1 \mod 5$\\
            $y_1 = -2$\\
            \\
            $y_2M_2 \equiv 1 \mod m_1$\\
            $5y_2 \equiv 1 \mod 7$\\
            $y_2 = 3$\\
            
            For the simultaneous congruence, $x$ can be expressed as: 
            \begin{align*}
                x &= \sum_{i=1}^{3}a_iy_iM_i\\
                &= (1 \cdot -2 \cdot 7)+(1 \cdot 3 \cdot 5)\\
                &= -14 + 15\\
                &= 1
            \end{align*}
            
            Therefore, $31^{30^{45}} \mod 35 = 1$
        \end{homeworkProblem}
\end{document}