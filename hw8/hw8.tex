\documentclass{article}

        \usepackage{fancyhdr}
        \usepackage{extramarks}
        \usepackage{amsmath}
        \usepackage{amssymb}
        \usepackage{amsthm}
        \usepackage{amsfonts}
        \usepackage{tikz}
        \usepackage[plain]{algorithm}
        \usepackage{algpseudocode}
        \usepackage{listings}
        \definecolor{mygreen}{rgb}{0,0.6,0}
        \definecolor{mygray}{rgb}{0.5,0.5,0.5}
        \definecolor{mymauve}{rgb}{0.58,0,0.82}

        \lstdefinelanguage{JavaScript}{
            keywords={typeof, new, true, false, catch, function, return, null, catch, switch, var, if, in, while, do, else, case, break},
            keywordstyle=\color{blue}\bfseries,
            ndkeywords={class, export, boolean, throw, implements, import, this},
            ndkeywordstyle=\color{darkgray}\bfseries,
            identifierstyle=\color{black},
            sensitive=false,
            comment=[l]{//},
            morecomment=[s]{/*}{*/},
            commentstyle=\color{purple}\ttfamily,
            stringstyle=\color{red}\ttfamily,
            morestring=[b]',
            morestring=[b]"
        }
        \lstset{
            language=JavaScript,
            backgroundcolor=\color{white},
            extendedchars=true,
            basicstyle=\footnotesize\ttfamily,
            showstringspaces=false,
            showspaces=false,
            numbers=left,
            numberstyle=\footnotesize,
            numbersep=9pt,
            tabsize=2,
            breaklines=true,
            showtabs=false,
            captionpos=b
        }
        
        \usetikzlibrary{automata,positioning}
        
        %
        % Basic Document Settings
        %
        
        \topmargin=-0.45in
        \evensidemargin=0in
        \oddsidemargin=0in
        \textwidth=6.5in
        \textheight=9.0in
        \headsep=0.25in
        
        \linespread{1.2}
        
        \pagestyle{fancy}
        \lhead{\hmwkAuthorName}
        \chead{\hmwkClass\ \hmwkTitle}
        \rhead{\firstxmark}
        \lfoot{\lastxmark}
        \cfoot{\thepage}
        
        \renewcommand\headrulewidth{0.4pt}
        \renewcommand\footrulewidth{0.4pt}
        
        \setlength\parindent{0pt}
        \setlength\parskip{0.2cm}
        
        %
        % Create Problem Sections
        %
        
        \newcommand{\enterProblemHeader}[1]{
            \nobreak\extramarks{}{Problem \arabic{#1} continued on next page\ldots}\nobreak{}
            \nobreak\extramarks{Problem \arabic{#1} (continued)}{Problem \arabic{#1} continued on next page\ldots}\nobreak{}
        }
        
        \newcommand{\exitProblemHeader}[1]{
            \nobreak\extramarks{Problem \arabic{#1} (continued)}{Problem \arabic{#1} continued on next page\ldots}\nobreak{}
            \stepcounter{#1}
            \nobreak\extramarks{Problem \arabic{#1}}{}\nobreak{}
        }
        
        \setcounter{secnumdepth}{0}
        \newcounter{partCounter}
        \newcounter{homeworkProblemCounter}
        \setcounter{homeworkProblemCounter}{1}
        \nobreak\extramarks{Problem \arabic{homeworkProblemCounter}}{}\nobreak{}
        
        %
        % Homework Problem Environment
        %
        % This environment takes an optional argument. When given, it will adjust the
        % problem counter. This is useful for when the problems given for your
        % assignment aren't sequential. See the last 3 problems of this template for an
        % example.
        %
        \newenvironment{homeworkProblem}[1][-1]{
            \ifnum#1>0
                \setcounter{homeworkProblemCounter}{#1}
            \fi
            \section{Problem \arabic{homeworkProblemCounter}}
            \setcounter{partCounter}{1}
            \enterProblemHeader{homeworkProblemCounter}
        }{
            \exitProblemHeader{homeworkProblemCounter}
        }
        
        %
        % Homework Details
        %   - Title
        %   - Due date
        %   - Class
        %   - Author
        %
        
        \newcommand{\hmwkTitle}{Homework\ \#8}
        \newcommand{\hmwkDueDate}{March 30, 2018}
        \newcommand{\hmwkClass}{CS 4823}
        \newcommand{\hmwkAuthorName}{\textbf{Christopher Tse}}
        
        %
        % Title Page
        %
        
        \title{
            \vspace{2in}
            \textmd{\textbf{\hmwkClass:\ \hmwkTitle}}\\
            \normalsize\vspace{0.1in}\small{Due\ on\ \hmwkDueDate}\\
            \vspace{3in}
        }
        
        \author{\hmwkAuthorName}
        \date{}
        
        \renewcommand{\part}[1]{\textbf{\large Part \Alph{partCounter}}\stepcounter{partCounter}\\}
        
        %
        % Various Helper Commands
        %
        
        % Useful for algorithms
        \newcommand{\alg}[1]{\textsc{\bfseries \footnotesize #1}}
        
        % For derivatives
        \newcommand{\deriv}[1]{\frac{\mathrm{d}}{\mathrm{d}x} (#1)}
        
        % For partial derivatives
        \newcommand{\pderiv}[2]{\frac{\partial}{\partial #1} (#2)}
        
        % Integral dx
        \newcommand{\dx}{\mathrm{d}x}
        
        % Alias for the Solution section header
        \newcommand{\solution}{\textbf{\large Solution}}
        
        % Probability commands: Expectation, Variance, Covariance, Bias
        \newcommand{\E}{\mathrm{E}}
        \newcommand{\Var}{\mathrm{Var}}
        \newcommand{\Cov}{\mathrm{Cov}}
        \newcommand{\Bias}{\mathrm{Bias}}
        \newcommand{\Z}{\mathbb{Z}}
        
        \begin{document}
        
        \maketitle
        
        \pagebreak
        
        \begin{homeworkProblem}
            Suppose that a Hill cipher with alphabet {0,1} and block length 3 is used to encrypt messages.  And suppose that we discover three plaintext-ciphtertext pairs:
            
            \[ (100) \rightarrow (101),\ (110) \rightarrow (110),\ (111) \rightarrow (001) \]
            
            Recover the encryption key.
            
            \textbf{Solution}
            
            Since Hill Ciphers are in the form $C = KP$ where C is ciphertext, K is the key, and P is the plaintext, we can set up a matrix equation with the given pairs as such:
            \[
                K
                \begin{pmatrix}
                    1 & 0 & 0\\
                    1 & 1 & 0\\
                    1 & 1 & 1
                \end{pmatrix}
                =
                \begin{pmatrix}
                    1 & 0 & 1\\
                    1 & 1 & 0\\
                    0 & 0 & 1
                \end{pmatrix}
            \]
            
            To find K, we must first ensure P is invertible. Since det(P) is not 0, it is invertible.
            
            We then determine the inverse of P, $P^{-1}$, which gives:
            
            \[
                P^{-1} = 
                \begin{pmatrix}
                    1 & 0 & 0\\
                    1 & 1 & 0\\
                    0 & 1 & 1
                \end{pmatrix}
            \]
            
            Since KP = C, then $CP^{-1} = K$
            
            \[
                \begin{pmatrix}
                    1 & 0 & 1\\
                    1 & 1 & 0\\
                    0 & 0 & 1
                \end{pmatrix}
                \cdotp
                \begin{pmatrix}
                    1 & 0 & 0\\
                    1 & 1 & 0\\
                    0 & 1 & 1
                \end{pmatrix}
                =
                \begin{pmatrix}
                    1 & 0 & 1\\
                    0 & 1 & 1\\
                    1 & 1 & 1
                \end{pmatrix}
            \]
        \end{homeworkProblem}
        
        \pagebreak
        
        \begin{homeworkProblem}
            Explain why in the AES S-box, the hexadecimal number \texttt{0x93} is substituted by \texttt{0xdc}. Please show step-by-step calculations
            
            \textbf{Solution}
            
            To find the S-box substitutions, we first find the inverse of our desired value over GF(2) and find the binary representation.
            
            \[ \texttt{0x93} \rightarrow \texttt{0x6D} = \texttt{0b01101101} \]
            
            We then perform logical AND on this binary representation with the affine transformation matrix as follows:
            
            \begin{verbatim}
    Input  1 0 1 1 0 1 1 0 (LSB First)
    Row 0  1 0 0 0 1 1 1 1
    Bit 0  1 0 0 0 0 1 1 0 = 1
    
    Row 1  1 1 0 0 0 1 1 1
    Bit 1  1 0 0 0 0 1 1 0 = 1
    
    Row 2  1 1 1 0 0 0 1 1 
    Bit 2  1 0 1 0 0 0 1 0 = 1
    
    Row 3  1 1 1 1 0 0 0 1
    Bit 3  1 0 1 1 0 0 0 0 = 1
    
    Row 4  1 1 1 1 1 0 0 0
    Bit 4  1 0 1 1 0 0 0 0 = 1
    
    Row 5  0 1 1 1 1 1 0 0
    Bit 5  0 0 1 1 0 1 0 0 = 1
    
    Row 6  0 0 1 1 1 1 1 0 
    Bit 6  0 0 1 1 0 1 1 0 = 0
    
    Row 7  0 0 0 1 1 1 1 1
    Bit 7  0 0 0 1 0 1 1 0 = 1
            \end{verbatim}
            
            Writing the binary result with MSB First, we get \texttt{0b10111111}, or \texttt{0xbf}. Finally, we XOR this result with \texttt{0x63} to get the s-box substitution value:
            
            \[
                \texttt{0xbf} \mathbin{\oplus} \texttt{0x63} = \texttt{0xdc}
            \]
        \end{homeworkProblem}
        
        \pagebreak
        
        \begin{homeworkProblem}
            Suppose the current state matrix before the AES MixColumns transformation is
            
            \[
                \begin{pmatrix}
                    O & K & L & A\\
                    H & O & M & A\\
                    I & L & L & I\\
                    N & O & I & S
                \end{pmatrix}
            \]
            
            (each letter is encoded as a byte according to the ASCII table), write a program to calculate the output state after the MixColumns transformation.
            
            \textbf{Solution}
            
            Solution begins on next page.
            
            \pagebreak
            
            \begin{lstlisting}[language=python]
// mixcolumns.sage
from sage.crypto.mq.rijndael_gf import RijndaelGF
from sage.crypto.util import bin_to_ascii

def my_mix_columns(string):
    """
    Takes an input string and performs AES
    MixColumns transform on it
    
    Arguments:
        string {str} -- Input string
        
    Throws:
        ValueError -- Input string must be 16 characters long
    """
    if len(string) != 16:
        raise ValueError("Input string must be 16 characters long")
        return
        
    rgf = RijndaelGF(4, 4)
    
    s = HexadecimalStrings().encoding(string)
    
    state = rgf._hex_to_GF(str(s))
    result = rgf.mix_columns(state)
    line = rgf._GF_to_hex(result)
    
    n = 2
    split = [line[i:i+n] for i in range(0, len(line), n)]
    
    def parse_and_mod(n):
        n = int(n, base=16)
        hexval = str(hex(n))
        return bin_to_ascii(Integer(int(hexval, base=16)).binary().zfill(8))
        
        
    
    encoded = "".join(map(parse_and_mod, split))
    print encoded      
            \end{lstlisting}
            
            \begin{verbatim}
sage: my_mix_columns("OKLAHOMAILLINOIS")
NL_TM@^XCLFIWXfr
            \end{verbatim}
            
            Therefore the resulting matrix would be
            
            \[
                \begin{pmatrix}
                    N & L & \_ & T\\
                    M & @ & \hat{} & X\\
                    C & L & F & I\\
                    W & X & f & r
                \end{pmatrix}
            \]
        \end{homeworkProblem}
\end{document}